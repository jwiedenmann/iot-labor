%!TEX root = ../../main.tex

\chapter{Introduction}

The aim of this project is to derive as much insight as possible from the sensor data collected during various motorcycle rides and evaluate my riding performance. The data set consists of approximately 500,000 data points, recorded using an \ac{IMU}. The sensors in use include an accelerometer, gyroscope, magnetometer, and temperature sensor, capturing detailed motion and environmental information throughout the rides.

The focus of this project is particularly on the accelerometer data. I aim to assess my motorcycle riding skills by analyzing how close I come to the maximum possible g-forces during turns, which is a key indicator of riding performance and cornering capability. This analysis will allow me to make informed statements about my riding precision and whether I am optimizing my handling in terms of the forces experienced during turns.

The data collection occurred during multiple motorcycle rides, and the sensor data was recorded continuously. To process and analyze this large dataset, I used Python along with well-established libraries suitable for handling big data, specifically Pandas for data manipulation and Matplotlib for visualization. These tools enabled me to clean, transform, and visualize the data to gain a comprehensive understanding of my riding patterns.

By focusing on the accelerometer data, I aim to develop a deeper understanding of my riding technique, particularly in cornering, and identify areas for improvement based on quantitative analysis.

