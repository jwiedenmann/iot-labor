%!TEX root = ../../main.tex

\chapter{Correlation and relationship analysis}

The objective of this chapter is to perform a correlation analysis between the temperature and other sensor readings collected during the motorcycle rides. The analysis aims to determine whether there is any significant relationship between the changes in temperature and the forces measured by the accelerometer, gyroscope, magnetometer, and the elapsed time (millis). Understanding these relationships (or lack thereof) is essential for evaluating whether temperature has an impact on the dynamic behavior of the motorcycle.

\section{Correlation Results Overview}

To explore the relationships, we calculated the Pearson correlation coefficient between the temperature and the values from each sensor. The Pearson correlation coefficient is a measure of the linear relationship between two variables, with values ranging from -1 (perfect negative correlation) to +1 (perfect positive correlation). A value close to 0 indicates no linear correlation.

The resulting correlation coefficients are displayed in Figure \ref{fig:cor:correlation_plot}.

\begin{itemize}
    \item \textbf{Temperature Correlation:} As expected, temperature exhibits a perfect correlation with itself (1.0), confirming the consistency of the temperature sensor over time.

    \item \textbf{Accelerometer Correlations:} The accelerometer readings (accX, accY, accZ) show no significant correlation with the temperature. The correlation coefficients are close to zero for all axes, indicating that the temperature changes are independent of the forces acting on the motorcycle in the forward/backward (accX), lateral (accY), and vertical (accZ) directions.

    \item \textbf{Gyroscope Correlations:} The gyroscope readings (gyrX, gyrY, gyrZ) also show no meaningful correlation with the temperature, with coefficients near zero. This suggests that the rotational forces, such as roll (gyrZ) or yaw (gyrX), do not have any linear relationship with the changes in ambient temperature.

    \item \textbf{Magnetometer Correlations:} The magnetometer data (magX, magY, magZ) shows only weak correlations with temperature, the highest being for magX with a coefficient of around 0.2. This suggests a small, non-significant relationship between magnetic field strength and temperature, possibly due to the sensor’s proximity to heat sources like the exhaust. However, this relationship is too weak to indicate any strong linear dependency.

    \item \textbf{Timestamps (millis) Correlation:} The correlation between temperature and the millis (elapsed time) is very low, indicating that while temperature increases over time, it does not have a simple linear relationship with the passage of time. This makes sense, as the temperature changes are primarily driven by environmental factors (e.g., moving the motorcycle outside, the effect of exhaust heat) rather than time alone.
\end{itemize}

\begin{figure}[h]
    \centering
    \includegraphics[height=.55\textwidth]{images/05/Correlation.png}
    \caption{Correlation data between the temperature and the other data columns.}
    \label{fig:cor:correlation_plot}
\end{figure}

\section{Interpretation of Results}

The correlation analysis reveals that temperature does not have any significant linear relationship with the dynamic forces measured by the accelerometer, gyroscope, or magnetometer. This means that, within the scope of this analysis, temperature changes are independent of the forces and movements of the motorcycle.

The lack of correlation between temperature and other sensor readings can be explained by the nature of the factors driving temperature changes. The temperature variations recorded during the ride are primarily influenced by environmental transitions—such as moving from a garage to a hot outdoor environment—and the proximity to the exhaust system. These changes are unrelated to the dynamic forces acting on the motorcycle, such as acceleration, braking, and cornering forces.

The accelerometer, gyroscope, and magnetometer sensors capture the mechanical behavior of the motorcycle, driven by rider input, road conditions, and riding dynamics. These factors operate independently of the temperature changes, which are largely driven by external environmental conditions and the heat generated by the exhaust system.